\documentclass[11pt, a4paper]{amsart}
\usepackage{amsrefs}

% Language setting
% Replace `english' with e.g. `spanish' to change the document language
\usepackage[english]{babel}

% Set page size and margins
% Replace `letterpaper' with`a4paper' for UK/EU standard size
%\usepackage[a4paper,top=2cm,bottom=2cm,left=3cm,right=3cm,marginparwidth=1.75cm]{geometry}

% The following causes equations to be numbered within sections
\numberwithin{equation}{section}
% We’ll use the equation counter for all our theorem environments, so
% that everything will be numbered in the same sequence.
%plain
%Theorem, Lemma, Corollary, Proposition, Conjecture,Criterion, Assertion
%definition
%Definition, Condition, Problem, Example, Exercise,Algorithm, Question, Axiom, Property, Assumption,Hypothesis
%remark
%Remark, Note, Notation, Claim, Summary,Acknowledgment, Case, Conclusion
%       Theorem environments
\theoremstyle{plain} %% This is the default, anyway
\newtheorem{thm}{Theorem}[section]
\newtheorem{cor}{Corollary}[section]
\newtheorem{lem}{Lemma}[section]
\newtheorem{prop}{Proposition}[section]
\newtheorem{conj}{Conjecture}[section]
\newtheorem{cri}{Criterion}[section]
\newtheorem{asser}{Assertion}[section]

\theoremstyle{definition}
\newtheorem{defn}{Definition}[section]
\newtheorem{ex}{Example}[section]
\newtheorem{cond}{Condition}[section]
\newtheorem{prob}{Problem}[section]
\newtheorem{exer}{Exercise}[section]
\newtheorem{ques}{Question}[section]
\newtheorem{axi}{Axiom}[section]
\newtheorem{fact}{Fact}[section]
\newtheorem{property}{Property}[section]
\newtheorem{hyp}{Hypothesis}[section]
\newtheorem{alg}{Algorithm}[section]
\newtheorem{assump}{Assumption}[section]

\theoremstyle{remark}
\newtheorem{rem}{Remark}[section]
\newtheorem{notation}{Notation}[section]
\newtheorem{terminology}{Terminology}[section]
\newtheorem{note}{Note}[section]
\newtheorem{claim}{Claim}[section]
\newtheorem{summary}{Summary}[section]
\newtheorem{case}{Case}[section]
\newtheorem{acknow}{Acknowledgment}[section]

% Useful packages
\usepackage{amsmath}
\usepackage{amsfonts}
\usepackage{amssymb}
\usepackage{amsthm}
\usepackage{graphicx}
\usepackage[colorlinks=true, allcolors=blue]{hyperref}
\usepackage{blindtext}
\usepackage{multicol}

\title{Lower level programming\\Exercises}
\author{Clóvis W. Bertholini Sb.}

\begin{document}

\maketitle
Bellow we have the questions and the answers about low level programming, Its are in the book and the git hub from Apress \cite{zhirkov} and \cite{gitzhirkov}.\\
\\
\textbf{Question 1)}\\
\\
It is time to make a first research based on the documentation [Intel 64 and IA-32 Architectures Software Developer's Manual]. Refer to the section 3.4.3 of the first volume to learn about register rflags. What is the meaning of flags CF, AF, ZF, OF, SF? What is the difference between OF and CF?\\
\\
\textbf{Answer:}\\
\\
\textbf{(a) Meaning of flags}\\
\\
3.4.3.1  Status Flags: The status flags (bits 0, 2, 4, 6, 7, and 11) of the EFLAGS register indicate the results of arithmetic instructions, such as the ADD, SUB, MUL, and DIV instructions. The status flag functions are:\\
\\
\textbf{CF} (bit 0) Carry flag — Set if an arithmetic operation generates a carry or a borrow out of the most-significant bit of the result; cleared otherwise. This flag indicates an overflow condition for unsigned-integer arithmetic. It is also used in multiple-precision arithmetic.\\
\\
\textbf{PF} (bit 2) Parity flag — Set if the least-significant byte of the result contains an even number of 1 bits; cleared otherwise.\\
\\
\textbf{AF} (bit 4) Auxiliary Carry flag — Set if an arithmetic operation generates a carry or a borrow out of bit 3 of the result; cleared otherwise. This flag is used in binary-coded decimal (BCD) arithmetic.\\
\\
\textbf{ZF} (bit 6) Zero flag — Set if the result is zero; cleared otherwise.\\
\\
\textbf{SF} (bit 7) Sign flag — Set equal to the most-significant bit of the result, which is the sign bit of a signed integer. (0 indicates a positive value and 1 indicates a negative value.)\\
\\
\textbf{O}F (bit 11) Overflow flag — Set if the integer result is too large a positive number or too small a negative number (excluding the sign-bit) to fit in the destination operand; cleared otherwise. This flag indicates an overflow condition for signed-integer (two’s complement) arithmetic.\\
\\
\textbf{(b) Difference between OF and CF}\\
\\
If the result of an arithmetic operation is treated as an unsigned integer, the CF flag indicates an out-of-range condition (carry or a borrow); if treated as a signed integer (two’s complement number), the OF flag indicates a carry or borrow. In other words, CF works with unsigned integers and OF works with signed integers.\\
\\
\textbf{Question 2)}\\
\\
What are the key principles of von Neumann architecture?\\
\\
\textbf{Answer:}\\
\begin{itemize}
    \item Memory stores only bits (a unit of information, a value equal 0 or 1).
    \item Memory stores both encoded instructions and data to operate on. There are no means to distinguish data from code: both are in fact bit strings.
    \item Memory is organized into cells, which are labeled with their respective indices in a natural way (for example, the cell \#43 follows the cell \#42). The indices start at 0. Cell size may vary (Von Neumann thought that each bit should have its address). Moderns computers take one byte (eight bits) as a memory cell size. So, the 0-th byte holds the first eight bits os the memory.
    \item The program consists of instructions that are fetched one after another. Their execution is sequential unless a special jump instruction is executed.
\end{itemize}





\begin{bibdiv}
        \begin{biblist}
            \bib{zhirkov}{book}{
                author={Lúcia A. Kinoshita},
                title={Programação em Baixo Nível},
                note={Traduced from english language: Low level Programming, by Igor Zhirkov},
                date={2018},
                publisher={Novatec},
                address={São Paulo},
                isbn={978-85-7522-667-4}
            }
            \bib{gitzhirkov}{webpage}{
                author={Igor Zhirkov},
                title={Apress/low-level-programming},
                accessdate={2021-7-11},
                url={https://github.com/Apress/low-level-programming}
            }
        \end{biblist}
    \end{bibdiv}

\end{document}
